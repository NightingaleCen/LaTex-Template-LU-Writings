%--------------------
% Packages
% -------------------
\documentclass[11pt,a4paper]{article}
\usepackage{LUwritings}
%-----------------------
% Set pdf information and add title, fill in the fields
%-----------------------
\hypersetup{
pdfsubject = {The template for writings in Lund University},
pdftitle = {Template for LU Writings},
pdfauthor = {John Smith}
}

%-----------------------
% Begin document
%-----------------------
\begin{document}

\title{LU Template}
\author{John Smith}

\maketitle

\section{Section}

This is a section example.\footnote{An example for a footnote} The following sections demonstrate some common elements you can use in your documents, including code snippets, figures, etc.

\subsection{Subsection}

This is a subsection that provides more detailed information. You can organize your content hierarchically using subsections and subsubsections.

\subsubsection{Subsubsection}

Here is a subsubsection with even more specific content. Below is an example of an unordered list:

\begin{itemize}
	\item First item in the list
	\item Second item in the list
	\item Third item in the list
\end{itemize}

\subsection{Another Subsection}

This subsection demonstrates an ordered list:

\begin{enumerate}
	\item First numbered item
	\item Second numbered item
	\item Third numbered item
\end{enumerate}

\subsubsection{Nested Lists}

You can also create nested lists for more complex structures:

\begin{itemize}
	\item Main item one
		\begin{itemize}
			\item Nested item one
			\item Nested item two
		\end{itemize}
	\item Main item two
\end{itemize}

\section{Figures}

This subsection demonstrates how to include images in your document. Fig.~\ref{fig:single_image} shows a single image example with the Lund University logo. The image is centered and scaled to 60\% of the text width for proper display.

\begin{figure}[H]
	\centering
	\includegraphics[width=0.6\textwidth]{img/Lund_university_L_RGB.png}
	\caption{Single Image Example}
	\label{fig:single_image}
\end{figure}

When you need to display multiple related images side by side, subfigures provide an excellent solution. Fig.~\ref{fig:subfigures} presents two subfigures: Fig.~\ref{fig:subfig1} displays the Lund University C logo on the left, while Fig.~\ref{fig:subfig2} shows the LTH logo on the right. Each subfigure can be referenced individually while also being part of a larger figure group.

\begin{figure}[H]
	\centering
	\begin{subfigure}{0.45\textwidth}
		\centering
		\includegraphics[width=\textwidth]{img/LundUniversity_C_RGB.png}
		\caption{First subfigure}
		\label{fig:subfig1}
	\end{subfigure}
	\hfill
	\begin{subfigure}{0.45\textwidth}
		\centering
		\includegraphics[width=\textwidth]{img/LTH_RGB_ENG.png}
		\caption{Second subfigure}
		\label{fig:subfig2}
	\end{subfigure}
	\caption{Two Subfigures Example}
	\label{fig:subfigures}
\end{figure}


\section{Tables}
This section demonstrates how to create professional-looking tables using the \texttt{booktabs} package. Table~\ref{tab:example} shows a simple table with proper formatting using horizontal rules.

\begin{table}[H]
	\centering
    \caption{Example Table with booktabs Formatting}
	\begin{tabular}{lcc}
		\toprule
		\textbf{Column 1} & \textbf{Column 2} & \textbf{Column 3} \\
		\midrule
		Row 1, Col 1 & 10 & 20 \\
		Row 2, Col 1 & 30 & 40 \\
		Row 3, Col 1 & 50 & 60 \\
		\bottomrule
	\end{tabular}
	\label{tab:example}
\end{table}


\section{Code snippets}

Below is an example of how to include and format code in your document. Code~\ref{code:code_example} demonstrates basic syntax highlighting and code formatting capabilities.

\begin{lstlisting}[language=Python, caption={Code Example}, label={code:code_example}]
def hello_world():
    print("hello world")
\end{lstlisting}

\section{Mathematics}

This section demonstrates various ways to typeset mathematical equations in \LaTeX{}. From simple inline equations to complex multi-line displays, these examples showcase the capabilities of the \texttt{amsmath} package.

\subsection{Inline Equations}

Inline equations are placed within text, such as $E = mc^2$, the famous mass-energy equivalence formula. Another example is the quadratic formula: $x = \frac{-b \pm \sqrt{b^2 - 4ac}}{2a}$.

\subsection{Display Equations}

A simple display equation is shown below:
\[
	\int_0^\infty e^{-x^2} \, dx = \frac{\sqrt{\pi}}{2}
\]

\subsection{Numbered Equations}

For equations that need to be referenced, use the \texttt{equation} environment:
\begin{equation}
	\label{eq:pythagorean}
	a^2 + b^2 = c^2
\end{equation}

The Pythagorean theorem is shown in Eq.~\ref{eq:pythagorean}.

\subsection{Multi-line Equations}

For systems of equations or derivations, the \texttt{align*} environment is useful:
\begin{align*}
	(x + y)^2 &= x^2 + 2xy + y^2 \\
	&= x^2 + xy + xy + y^2 \\
	&= x(x + y) + y(x + y)
\end{align*}

With numbering and references:
\begin{align}
	f(x) &= x^2 + 3x + 2 \label{eq:quadratic} \\
	f'(x) &= 2x + 3 \label{eq:derivative}
\end{align}

From Eq.~\ref{eq:quadratic} and Eq.~\ref{eq:derivative}, we can analyze the function's behavior.

\subsection{Cases and Piecewise Functions}

\begin{equation}
	|x| = \begin{cases}
		x & \text{if } x \geq 0 \\
		-x & \text{if } x < 0
	\end{cases}
\end{equation}

\end{document}